\documentclass{article}

\usepackage[utf8]{inputenc}
\usepackage[T1]{fontenc}
\usepackage[english]{babel}
\usepackage[margin=1in]{geometry}
\usepackage{fancyhdr}
\usepackage[hidelinks]{hyperref}
\usepackage{tikz}
\usepackage{fancyhdr}
\usepackage{float}
\usepackage{listings}
\usetikzlibrary{trees}
\author{R\'emy Detobel \& Nathan Liccardo}
\title{Report Project 3~: XQuery}

\usepackage{color}
\definecolor{Maroon}{rgb}{0.5,0,0}

\lstdefinelanguage{XML}
{
  basicstyle=\ttfamily\footnotesize,
  morestring=[b]",
  moredelim=[s][\bfseries\color{Maroon}]{<}{\ },
  moredelim=[s][\bfseries\color{Maroon}]{</}{>},
  moredelim=[l][\bfseries\color{Maroon}]{/>},
  moredelim=[l][\bfseries\color{Maroon}]{>},
  morecomment=[s]{<?}{?>},
  morecomment=[s]{<!--}{-->},
  commentstyle=\color{DarkOliveGreen},
  stringstyle=\color{blue},
  identifierstyle=\color{red},
  showstringspaces=false
}

\renewcommand\thesection{\arabic{section}}

\begin{document}
\maketitle

\section{Introduction}
  This document aims to detail each choices and hypothesis we made during our implementation. As a reminder, we were assigned to implement three different XQuery programs. Each of those query use a part of the BDLP database (DBLP-excerpt). The structure of this report will be divided into three parts (one per program).

\section{First XQuery program}
  For this first program, we were assigned (for each author) to return the number of co-authors and the number of joint publications with each of them. To realise this, we start by iterate on each author as below~: 
  \begin{itemize}
    \item \begin{verbatim}
for $author in //author
    return <author>
        ....
    </author>\end{verbatim}
  \end{itemize}
  Which will create an \verb|<author> ... </author>| bloc for each author. Inside each of these blocs, we will find following informations (in sequence).
  
  \subsection{Author name and Co-authors informations}
    The author name and the number of co-authors are respectively obtained using \verb|data| and \verb|count| functions (defined by XQuery). Below instructions are formal definitions of \verb|name| and \verb|coauthors| tags~:
    \begin{itemize}
      \item \verb|<name>{data($author)}</name>|
      \item \verb|<coauthors number="{count(//*[author=$author]/author)-1}"> ... </coauthors>|
    \end{itemize}
    
    \subsubsection{Co-authors informations}
      For each co-author, we must obtain his name and the number of joint publications. Both informations will be contained inside \verb|<coauthor> ... </coauthor>| tags (itself contained in the \verb|coauthors| bloc). Each of the \verb|coauthor| blocs are created by iterating on co-authors as below~:
      \begin{itemize}
	\item \begin{verbatim}
for $coauthor in //*[author=$author]/author[not(.=$author)]
    return <coauthor>
    ....
    </coauthor>\end{verbatim}
      \end{itemize}
      
      \paragraph{Co-author name}
	As for the author name, co-author name is reached using the \verb|data| function on the \verb|coauthor| variable. This is realised as follow~:
	\begin{itemize}
	  \item \verb|<name>{data($coauthor)}</name>|
	\end{itemize}
	
      \paragraph{Joint publications}
	For each co-author, we must retrieve the number of joint publications. Once again, this is achieved using the \verb|count| function available in XQuery. Here is the final instruction~:
	\begin{itemize}
	  \item \verb|<nb_joint_pubs>{count(//*[author=$author]/author[.=$coauthor])}</nb_joint_pubs>|
	\end{itemize}


\section{Second XQuery program}
  % Pour cette seconde requête, nous devons lié chaque ``proceeding'' a ses ``inproceeding''. Pour cela, on utilise l'attribut ``key'' définit dans la balise \verb|<proceedings>| et la valeur définie par les balises \verb|<crossref>|.
  For this second request, we must link each \textit{proceeding} to its \textit{inproceeding}. To do this, we use the ``key'' attribute defined in the tag \verb|<proceedings>| and the value defined by the tags \verb|<crossref>|.
  
  % Tout d'abord, on extrait le titre de chaque proceeding~:
  First, the title of each proceeding is extracted~:
  \begin{itemize}
    \item \begin{verbatim}
for $proceeding in //proceedings
    return <proceedings>
                <proc_title>{data($proceeding/title)}</proc_title>
                ...
            </proceedings>\end{verbatim}
  \end{itemize}
  
  % On parcourt ensuite chaque inproceeding où la valeur se trouvant entre les balises \verb|crossref| correspond à la valeur définie dans l'attribut ``key'' de la proceeding actuelle.
  Each inproceedinging is then browsed where the value between the \verb|crossref| tags corresponds to the value defined in the ``key'' attribute of the current procedure~:
  \begin{itemize}
    \item \begin{verbatim}
for $inproceeding in //inproceedings[crossref=data($proceeding/@key)]
  return <title>{data($inproceeding/title)}</title>\end{verbatim}
  \end{itemize}
  

\section{Third XQuery program}




\end{document}

